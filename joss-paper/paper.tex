% Options for packages loaded elsewhere
\PassOptionsToPackage{unicode}{hyperref}
\PassOptionsToPackage{hyphens}{url}
%
\documentclass[
]{article}
\usepackage{amsmath,amssymb}
\usepackage{lmodern}
\usepackage{iftex}
\ifPDFTeX
  \usepackage[T1]{fontenc}
  \usepackage[utf8]{inputenc}
  \usepackage{textcomp} % provide euro and other symbols
\else % if luatex or xetex
  \usepackage{unicode-math}
  \defaultfontfeatures{Scale=MatchLowercase}
  \defaultfontfeatures[\rmfamily]{Ligatures=TeX,Scale=1}
\fi
% Use upquote if available, for straight quotes in verbatim environments
\IfFileExists{upquote.sty}{\usepackage{upquote}}{}
\IfFileExists{microtype.sty}{% use microtype if available
  \usepackage[]{microtype}
  \UseMicrotypeSet[protrusion]{basicmath} % disable protrusion for tt fonts
}{}
\makeatletter
\@ifundefined{KOMAClassName}{% if non-KOMA class
  \IfFileExists{parskip.sty}{%
    \usepackage{parskip}
  }{% else
    \setlength{\parindent}{0pt}
    \setlength{\parskip}{6pt plus 2pt minus 1pt}}
}{% if KOMA class
  \KOMAoptions{parskip=half}}
\makeatother
\usepackage{xcolor}
\usepackage[margin=1in]{geometry}
\usepackage{graphicx}
\makeatletter
\def\maxwidth{\ifdim\Gin@nat@width>\linewidth\linewidth\else\Gin@nat@width\fi}
\def\maxheight{\ifdim\Gin@nat@height>\textheight\textheight\else\Gin@nat@height\fi}
\makeatother
% Scale images if necessary, so that they will not overflow the page
% margins by default, and it is still possible to overwrite the defaults
% using explicit options in \includegraphics[width, height, ...]{}
\setkeys{Gin}{width=\maxwidth,height=\maxheight,keepaspectratio}
% Set default figure placement to htbp
\makeatletter
\def\fps@figure{htbp}
\makeatother
\setlength{\emergencystretch}{3em} % prevent overfull lines
\providecommand{\tightlist}{%
  \setlength{\itemsep}{0pt}\setlength{\parskip}{0pt}}
\setcounter{secnumdepth}{-\maxdimen} % remove section numbering
\newlength{\cslhangindent}
\setlength{\cslhangindent}{1.5em}
\newlength{\csllabelwidth}
\setlength{\csllabelwidth}{3em}
\newlength{\cslentryspacingunit} % times entry-spacing
\setlength{\cslentryspacingunit}{\parskip}
\newenvironment{CSLReferences}[2] % #1 hanging-ident, #2 entry spacing
 {% don't indent paragraphs
  \setlength{\parindent}{0pt}
  % turn on hanging indent if param 1 is 1
  \ifodd #1
  \let\oldpar\par
  \def\par{\hangindent=\cslhangindent\oldpar}
  \fi
  % set entry spacing
  \setlength{\parskip}{#2\cslentryspacingunit}
 }%
 {}
\usepackage{calc}
\newcommand{\CSLBlock}[1]{#1\hfill\break}
\newcommand{\CSLLeftMargin}[1]{\parbox[t]{\csllabelwidth}{#1}}
\newcommand{\CSLRightInline}[1]{\parbox[t]{\linewidth - \csllabelwidth}{#1}\break}
\newcommand{\CSLIndent}[1]{\hspace{\cslhangindent}#1}
\ifLuaTeX
  \usepackage{selnolig}  % disable illegal ligatures
\fi
\IfFileExists{bookmark.sty}{\usepackage{bookmark}}{\usepackage{hyperref}}
\IfFileExists{xurl.sty}{\usepackage{xurl}}{} % add URL line breaks if available
\urlstyle{same} % disable monospaced font for URLs
\hypersetup{
  pdftitle={Konfound: An R Sensitivity Analysis Package to Quantify the Robustness of Causal},
  hidelinks,
  pdfcreator={LaTeX via pandoc}}

\title{Konfound: An R Sensitivity Analysis Package to Quantify the
Robustness of Causal}
\author{}
\date{\vspace{-2.5em}24 July 2023}

\begin{document}
\maketitle

\hypertarget{quantifying-the-robustness-of-inferences}{%
\section{Quantifying the Robustness of
Inferences}\label{quantifying-the-robustness-of-inferences}}

Statistical methods which quantify the conditions necessary to alter
inferences are important to a variety of disciplines (Razavi et al.
2021). One line of work is rooted in linear models and foregrounds the
sensitivity of inferences to the strength of omitted variables (K. A.
Frank 2000; Cinelli 2019). A more recent approach is rooted in the
potential outcomes framework for causal inference and foregrounds how
hypothetical changes in a sample would alter an inference if such
variables or cases were otherwise observed {[}K. A. A. K.-S. M. Frank
(2007); G. A. A. Frank Kenneth A. AND Sykes (2008); S. J. A. D. Frank
Kenneth A. AND Maroulis (2013); Xu (2019)). We have implemented two
measures from these lines of work within R via the \texttt{konfound} R
package. One measure is the Impact Threshold of a Confounding Variable
(ITCV), which generates statements such as ``to invalidate an inference
of an effect, an omitted variable would have to be correlated at \_\_
with the predictor of interest and with the outcome'' (K. A. Frank
2000). This sensitivity analysis can be calculated for any linear model.
The second measure is the Robustness of an Inference to Replacement
(RIR) which generates statements such as ``to invalidate the inference,
\_\_ \% of the cases would have to be replaced with counterfactual cases
with zero effect of the treatment'' (S. J. A. D. Frank Kenneth A. AND
Maroulis 2013). The RIR represents a more generally applicable approach
not limited to linear models, and is recommended for all cases that use
binary outcomes (Q. A. M. Frank Kenneth A. AND Lin 2021).

\hypertarget{statement-of-need-the-need-for-an-r-package}{%
\section{Statement of Need: The Need for an R
Package}\label{statement-of-need-the-need-for-an-r-package}}

We have implemented these recent developments of sensitivity analysis
for causal inferences within R via the \texttt{konfound} R package. In
particular, the \texttt{konfound} package is used to calculate two
robustness indices: ITCV and RIR.

This paper provides an overview of two core functions within the
\texttt{konfound\ package}: \texttt{konfound()}, \texttt{pkonfound()}.
These functions allow users to calculate the robustness of inferences
using a model estimated in R in or using information about a model from
a published study, respectively. The package is available from the
Comprehensive R Archive Network (CRAN) at
\url{https://CRAN.R-project.org/package=konfound}; it can be installed
via the standard \texttt{install.packages(“konfound”)} function. These
two core functions within the \texttt{konfound} package are briefly
summarized below along with their functionality.

\hypertarget{summary-and-functionality}{%
\section{Summary and Functionality}\label{summary-and-functionality}}

\begin{enumerate}
\def\labelenumi{\arabic{enumi}.}
\tightlist
\item
  \textbf{konfound}: This function is used for fitted models and
  calculates the ITCV and RIR. This function currently works for linear
  models created with \texttt{lm()}, \texttt{glm()}, and \texttt{lmer()}
  fitted in R. The output printed in the R console is the bias that must
  be present and the number of cases that would have to be replaced with
  cases for which there is no effect to invalidate the inference.
\end{enumerate}

\hypertarget{example-for-linear-models-fit-with-lm}{%
\subsection{\texorpdfstring{\emph{Example for linear models fit with
lm()}}{Example for linear models fit with lm()}}\label{example-for-linear-models-fit-with-lm}}

For this example, we will use the \texttt{concord1} dataset built into
the \texttt{konfound} package. This dataset comes from Hamilton's (1983)
study which examines the causal mechanism behind household water
conservation in a U.S. city.

We will model the impact the following variables have on household water
consumption in 1981: household water consumption in 1980
(\texttt{water80}), household income (\texttt{income}), education level
of household survey respondent (\texttt{educat}), retirement status of
respondent (\texttt{retire}), and number of individuals in household in
1980 (\texttt{peop80}).

\begin{verbatim}
library(konfound)

## Sensitivity analysis as described in Frank, Maroulis, Duong, and Kelcey (2013) and in Frank (2000).
## For more information visit http://konfound-it.com.

m <- lm(water81 ~ water80 + income + educat + retire + peop80, data = concord1)
summary(m)

## 
## Call:
## lm(formula = water81 ~ water80 + income + educat + retire + peop80, 
##     data = concord1)
## 
## Residuals:
##     Min      1Q  Median      3Q     Max 
## -4035.5  -453.4   -62.7   384.2  4995.5 
## 
## Coefficients:
##             Estimate Std. Error t value Pr(>|t|)    
## (Intercept) 299.7437   210.0136   1.427  0.15414    
## water80       0.4943     0.0268  18.445  < 2e-16 ***
## income       22.6031     3.5023   6.454 2.62e-10 ***
## educat      -44.2578    13.4381  -3.293  0.00106 ** 
## retire      155.4727    96.3389   1.614  0.10721    
## peop80      225.1984    28.7048   7.845 2.73e-14 ***
## ---
## Signif. codes:  0 '***' 0.001 '**' 0.01 '*' 0.05 '.' 0.1 ' ' 1
## 
## Residual standard error: 864.1 on 490 degrees of freedom
## Multiple R-squared:  0.6653, Adjusted R-squared:  0.6619 
## F-statistic: 194.8 on 5 and 490 DF,  p-value: < 2.2e-16
\end{verbatim}

Results indicate that all variables except retire have a significant
effect on 1981 household water consumption.

\hypertarget{itcv-example-for-linear-models-fit-with-lm}{%
\subsection{\texorpdfstring{\emph{ITCV example for linear models fit
with
lm()}}{ITCV example for linear models fit with lm()}}\label{itcv-example-for-linear-models-fit-with-lm}}

Now let's examine the robustness of the \texttt{peop80} effect by
calculating the ITCV.

\begin{verbatim}
konfound(m, peop80, index = "IT")

## Impact Threshold for a Confounding Variable:
## The minimum impact of an omitted variable to invalidate an inference for a null hypothesis of 0 effect is based on a correlation of  0.52
##  with the outcome and at  0.52  with the predictor of interest (conditioning on observed covariates) based on a threshold of 
## 0.089 for statistical significance (alpha = 0.05).
## 
## Correspondingly the impact of an omitted variable (as defined in Frank 2000) must be 
## 0.52 X 0.52 = 0.27 to invalidate an inference for a null hypothesis of 0 effect.
## See Frank (2000) for a description of the method.
## 
## Citation:
## Frank, K. (2000). Impact of a confounding variable on the
## inference of a regression coefficient. Sociological Methods and Research, 29 (2), 147-194

## For more detailed output, consider setting `to_return` to table

## To consider other predictors of interest, consider setting `test_all` to TRUE.
\end{verbatim}

The output indicates that in order to invalidate the inference that
\texttt{peop80} has an effect on \texttt{water81} using statistical
significance as a threshold (e.g., p=.05), an omitted variable would
have to be correlated at 0.52 with \texttt{peop80} and 0.52 with
\texttt{water81}, conditioning on observed covariates. Correspondingly,
the impact of an omitted variable (as defined in (K. A. Frank 2000))
must be 0.52 X 0.52 = 0.27.

\hypertarget{rir-example-for-linear-models-fit-with-lm}{%
\subsection{\texorpdfstring{\emph{RIR example for linear models fit with
lm()}}{RIR example for linear models fit with lm()}}\label{rir-example-for-linear-models-fit-with-lm}}

We can also examine the robustness by calculating the RIR.

\begin{verbatim}
konfound(m, peop80, index = "RIR") 

## Robustness of Inference to Replacement (RIR):
## To invalidate an inference,  74.955 % of the estimate would have to be due to bias. 
## This is based on a threshold of 56.4 for statistical significance (alpha = 0.05).
## 
## To invalidate an inference,  372  observations would have to be replaced with cases
## for which the effect is 0 (RIR = 372).
## 
## See Frank et al. (2013) for a description of the method.
## 
## Citation: Frank, K.A., Maroulis, S., Duong, M., and Kelcey, B. (2013).
## What would it take to change an inference?
## Using Rubin's causal model to interpret the robustness of causal inferences.
## Education, Evaluation and Policy Analysis, 35 437-460.

## For more detailed output, consider setting `to_return` to table

## To consider other predictors of interest, consider setting `test_all` to TRUE.
\end{verbatim}

The output presents two interpretations of the RIR. First, 74.956\% of
the estimated effect of \texttt{peop80} on \texttt{water81} would have
to be attributed to bias to invalidate the inference. Equivalently, we
would expect to have to replace 372 out of the 486 households (about
76\%) with cases for which \texttt{peop80} had no effect to invalidate
the inference. See (S. J. A. D. Frank Kenneth A. AND Maroulis 2013; Q.
A. M. Frank Kenneth A. AND Lin 2021) for guidelines on evaluating the
RIR relative to the bias accounted for by observed covariates or
published norms.

\begin{enumerate}
\def\labelenumi{\arabic{enumi}.}
\setcounter{enumi}{1}
\tightlist
\item
  \textbf{pkonfound}: This function quantifies the sensitivity for
  analyses which have already been conducted, such as in an
  already-published study or from analysis carried out using other
  software. This function calculates 1) how much bias there must be in
  an estimate to invalidate/sustain an inference which can be
  interpreted as the percentage of cases that would have to be replaced
  (with cases for which the predictor had no effect on the outcome) to
  invalidate the inference and 2) the impact of an omitted variable
  necessary to invalidate/sustain an inference for a regression
  coefficient (note: impact is defined as the correlation between
  omitted variable and focal predictor x correlation between omitted
  variable and outcome).
\end{enumerate}

\hypertarget{itcv-example-for-a-regression-analysis}{%
\subsection{\texorpdfstring{\emph{ITCV example for a regression
analysis}}{ITCV example for a regression analysis}}\label{itcv-example-for-a-regression-analysis}}

For this example, the following parameters from a regression analysis
would be entered into the \texttt{pkonfound} function: unstandardized
beta coefficient for the predictor of interest (B = 2), standard error
(SE = .4), sample size (n = 100), and the number of covariates (cv = 3).

\begin{verbatim}
pkonfound(2, .4, 100, 3, index = "IT")

## Impact Threshold for a Confounding Variable:
## The minimum impact of an omitted variable to invalidate an inference for a null hypothesis of 0 effect is based on a correlation of  0.568
##  with the outcome and at  0.568  with the predictor of interest (conditioning on observed covariates) based on a threshold of 
## 0.201 for statistical significance (alpha = 0.05).
## 
## Correspondingly the impact of an omitted variable (as defined in Frank 2000) must be 
## 0.568 X 0.568 = 0.323 to invalidate an inference for a null hypothesis of 0 effect.
## See Frank (2000) for a description of the method.
## 
## Citation:
## Frank, K. (2000). Impact of a confounding variable on the
## inference of a regression coefficient. Sociological Methods and Research, 29 (2), 147-194

## For other forms of output, run ?pkonfound and inspect the to_return argument

## For models fit in R, consider use of konfound().
\end{verbatim}

The output indicates that in order to invalidate the inference that the
predictor of interest has a greater than zero effect, an omitted
variable would have to be correlated at 0.568 with the outcome and at
0.568 with the predictor of interest, conditioning on observed
covariates. Correspondingly, the impact of an omitted variable (ITCV)
must be 0.568 X 0.568 = 0.323 to invalidate an inference for a null
hypothesis of 0 effect.

\hypertarget{rir-example-for-a-regression-analysis}{%
\subsection{\texorpdfstring{\emph{RIR example for a regression
analysis}}{RIR example for a regression analysis}}\label{rir-example-for-a-regression-analysis}}

\begin{verbatim}
pkonfound(2, .4, 100, 3, index = "RIR")

## Robustness of Inference to Replacement (RIR):
## To invalidate an inference,  60.29 % of the estimate would have to be due to bias. 
## This is based on a threshold of 0.794 for statistical significance (alpha = 0.05).
## 
## To invalidate an inference,  60  observations would have to be replaced with cases
## for which the effect is 0 (RIR = 60).
## 
## See Frank et al. (2013) for a description of the method.
## 
## Citation: Frank, K.A., Maroulis, S., Duong, M., and Kelcey, B. (2013).
## What would it take to change an inference?
## Using Rubin's causal model to interpret the robustness of causal inferences.
## Education, Evaluation and Policy Analysis, 35 437-460.

## For other forms of output, run ?pkonfound and inspect the to_return argument

## For models fit in R, consider use of konfound().
\end{verbatim}

The output indicates that to invalidate the inference, approximately
60\% of the estimated effect of the predictor of interest would have to
be due to bias in this model to invalidate this inference. Equivalently,
60 out of 100 cases would have to be replaced with zero effect cases in
order to invalidate this inference. See (S. J. A. D. Frank Kenneth A.
AND Maroulis 2013; Q. A. M. Frank Kenneth A. AND Lin 2021) for
guidelines on evaluating the RIR relative to the bias accounted for by
observed covariates or published norms.

\hypertarget{doing-and-learning-more}{%
\section{Doing and Learning More}\label{doing-and-learning-more}}

We have created a Shiny interactive web application at
\url{http://konfound-it.com/} that can be applied to linear models, 2 x
2 tables resulting corresponding to treatment and control by success and
failure conditions, and logistic regression models (functionality for
designs including mediation, differences in difference, and regression
discontinuity are under development). We are also developing extensions
of the sensitivity analysis techniques described in this paper,
including preserving the standard error (Frank, Kenneth A. AND Lin,
Qinyun AND Xu, Ran AND Maroulis, Spiro AND Mueller, Anna 2023) and
calculating the coefficient of proportionality (Q. A. M. Frank Kenneth
A. AND Lin 2022) for ITCV analyses. Additional documentation on the R
package and future extensions will be available at
\url{https://konfound-project.github.io/konfound/}.

\hypertarget{acknowledgements}{%
\section{Acknowledgements}\label{acknowledgements}}

The research reported here was supported by the Institute of Education
Sciences, U.S. Department of Education, through Grant R305D220022 to
Michigan State University. The opinions expressed are those of the
authors and do not represent views of the Institute or the U.S.
Department of Education.

\hypertarget{references}{%
\section*{References}\label{references}}
\addcontentsline{toc}{section}{References}

\hypertarget{refs}{}
\begin{CSLReferences}{1}{0}
\leavevmode\vadjust pre{\hypertarget{ref-cinelli2019}{}}%
Cinelli, Chad, Carlos AND Hazlett. 2019. {``{Making Sense of
Sensitivity: Extending Omitted Variable Bias}.''} \emph{Journal of the
Royal Statistical Society Series B: Statistical Methodology} 82 (1):
39--67. \url{https://doi.org/10.1111/rssb.12348}.

\leavevmode\vadjust pre{\hypertarget{ref-frank2008}{}}%
Frank, Gary AND Anagnostopoulos, Kenneth A. AND Sykes. 2008. {``Does
NBPTS Certification Affect the Number of Colleagues a Teacher Helps with
Instructional Matters?''} \emph{Educational Evaluation and Policy
Analysis} 30 (1): 3--30. \url{http://www.jstor.org/stable/30128050}.

\leavevmode\vadjust pre{\hypertarget{ref-frank2000}{}}%
Frank, Kenneth A. 2000. {``Impact of a {Confounding} {Variable} on a
{Regression} {Coefficient}.''} \emph{Sociological Methods \& Research}
29 (2): 147--94. \url{https://doi.org/10.1177/0049124100029002001}.

\leavevmode\vadjust pre{\hypertarget{ref-frank2007}{}}%
Frank, Kenneth A. AND Kyung-Seok Min. 2007. {``10. Indices of Robustness
for Sample Representation.''} \emph{Sociological Methodology} 37 (1):
349--92. \url{https://doi.org/10.1111/j.1467-9531.2007.00186.x}.

\leavevmode\vadjust pre{\hypertarget{ref-frank2023}{}}%
Frank, Kenneth A. AND Lin, Qinyun AND Xu, Ran AND Maroulis, Spiro AND
Mueller, Anna. 2023. {``Quantifying the Robustness of Causal Inferences:
Sensitivity Analysis for Pragmatic Social Science.''} \emph{Social
Science Research} 110 (February).
\url{https://doi.org/10.1016/j.ssresearch.2022.102815}.

\leavevmode\vadjust pre{\hypertarget{ref-frank2021}{}}%
Frank, Qinyun AND Maroulis, Kenneth A. AND Lin. 2021. {``Hypothetical
Case Replacement Can Be Used to Quantify the Robustness of Trial
Results.''} \emph{Journal of Clinical Epidemiology} 134: 150--59.
\url{https://doi.org/10.1016/j.jclinepi.2021.01.025}.

\leavevmode\vadjust pre{\hypertarget{ref-frank2022}{}}%
---------. 2022. {``Improving Oster's δ*: Exact Calculation for the
Coefficient of Proportionality Without Subjective Specification of a
Baseline Model.''} \emph{SSRN}.
\url{https://papers.ssrn.com/sol3/papers.cfm?abstract_id=4305243}.

\leavevmode\vadjust pre{\hypertarget{ref-frank2013}{}}%
Frank, Spiro J. AND Duong, Kenneth A. AND Maroulis. 2013. {``What
{Would} {It} {Take} to {Change} an {Inference}? {Using} {Rubin}'s
{Causal} {Model} to {Interpret} the {Robustness} of {Causal}
{Inferences}.''} \emph{Educational Evaluation and Policy Analysis} 35
(4): 437--60. \url{https://doi.org/10.3102/0162373713493129}.

\leavevmode\vadjust pre{\hypertarget{ref-razavi2021}{}}%
Razavi, Saman, Anthony Jakeman, Andrea Saltelli, Clémentine Prieur,
Bertrand Iooss, Emanuele Borgonovo, Elmar Plischke, et al. 2021. {``The
{Future} of {Sensitivity} {Analysis}: {An} Essential Discipline for
Systems Modeling and Policy Support.''} \emph{Environmental Modelling \&
Software} 137 (March): 104954.
\url{https://doi.org/10.1016/j.envsoft.2020.104954}.

\leavevmode\vadjust pre{\hypertarget{ref-xu2019}{}}%
Xu, Kenneth A. AND Maroulis, Ran AND Frank. 2019. {``Konfound: Command
to Quantify Robustness of Causal Inferences.''} \emph{The Stata Journal}
19 (3): 523--50. \url{https://doi.org/10.1177/1536867X19874223}.

\end{CSLReferences}

\end{document}
